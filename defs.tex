% TO ADD: wreath products, characteristic subgroups, conjugation
\documentclass{article}
\usepackage{amsmath}
\usepackage{amssymb}
\usepackage{amsfonts}
\usepackage{mathtools}  % \coloneqq
\newcommand{\N}{\mathbb{N}}
\newcommand{\Z}{\mathbb{Z}}
\DeclareMathOperator{\lcm}{lcm}
\DeclareMathOperator{\Dih}{Dih}
\DeclareMathOperator{\Dic}{Dic}
\DeclareMathOperator{\Aut}{Aut}
\newcommand{\gen}[1]{\langle #1\rangle}
\newcommand{\normaleq}{\trianglelefteq}
\newcommand{\normal}{\triangleleft}
\title{Concepts \& Definitions Used in This Library}
\author{John T. Wodder II}
\begin{document}
\maketitle

A \emph{group} is a set $G$ together with a binary operation $\star\colon G\times G\to G$ such that:
\begin{itemize}
\item $\star$ is \emph{associative}: for all $x,y,z\in G$, $(x\star y)\star z = x\star(y\star z)$.
\item $\star$ has an \emph{identity}: there exists a (provably unique) $e\in G$ (the \emph{identity}) such that, for all $x\in G$, $e\star x = x\star e = x$.
\item $\star$ is \emph{invertible}: for each $x\in G$, there exists a (provably unique) $x'\in G$ (the \emph{inverse} of $x$) such that $x\star x' = x'\star x = e$.
\end{itemize}


\section{Notations \& Conventions}
\begin{itemize}
\item $\N$ is the set of all nonnegative integers (i.e., it includes 0).
\item The set of all nonnegative integers less than a positive integer $n$ (i.e., 0 through $n-1$) is here denoted $\N_n$.
\item The group operation for every group is here denoted $\cdot$, and $x\cdot y$ is further abbreviated to $xy$ for all $x,y$.
\item As group operations are associative, $(xy)z = x(yz)$ is here denoted $xyz$, and likewise for higher numbers of elements.
\item The identity for every group is here denoted $1$.
\item The inverse of an element $x$ of a group is here denoted $x^{-1}$.
\item For all elements $x$ of a group and positive integers $n$:
 \begin{itemize}
 \item $x^0 \coloneqq 1$
 \item $x^n \coloneqq \underbrace{x\cdot x\cdot\cdots\cdot x}_{n\ \mathrm{times}}$
 \item $x^{-n} \coloneqq (x^{-1})^n = (x^n)^{-1}$
 \end{itemize}
\item Given an element $x$ and subset $A$ of a group $G$, define $xA \coloneqq \{xa : a\in A\}$ and $Ax\coloneqq \{ax : a\in A\}$.
\item Given subsets $A,B$ of a group $G$, define $AB\coloneqq\{ab : a\in A, b\in B\}$.
\end{itemize}


\section{Definitions}
\begin{itemize}
\item The \emph{order of a group $G$}, denoted $|G|$, is the number of elements in $G$.  % What's the precise definition when $G$ is infinite?
\item The \emph{order of an element $x\in G$}, denoted $|x|$, is the smallest positive integer $n$ such that $x^n = 1$.  If there is no such positive integer (which is possible only when $G$ is infinite), $|x| = \infty$.
\item Two elements $x$ and $y$ of a group $G$ \emph{commute} iff $xy = yx$ (or, equivalently, $yxy^{-1} = y^{-1}xy = x$ or $xyx^{-1} = x^{-1}yx = y$).
\item A group $G$ is \emph{abelian} iff $xy = yx$ (or, equivalently, $y^{-1}xy = x$) for all $x,y\in G$.

\item A \emph{subgroup} of a group $G$ is a subset $H$ of $G$ that is also a group under $G$'s operation.  A \emph{proper subgroup} is a subgroup that is a proper subset.  The \emph{trivial subgroup}, denoted $1$, is the subgroup containing only the identity element.
 \begin{itemize}
 \item ``$H$ is a subgroup of $G$'' is denoted $H\leq G$.
 \item ``$H$ is a proper subgroup of $G$'' is denoted $H<G$.
 \end{itemize}
\item A \emph{normal subgroup} of a group $G$ is a subgroup $N$ of $G$ such that $gNg^{-1} = N$ for all $g\in G$.
 \begin{itemize}
 \item ``$H$ is a normal subgroup of $G$'' is denoted $H\normaleq G$.
 \item ``$H$ is a proper normal subgroup of $G$'' is denoted $H\normal G$.
 \end{itemize}
\item The \emph{closure} of or \emph{subgroup generated by} a subset $A$ of a group $G$, denoted $\gen{A}$, is the intersection of all subgroups of $G$ that contain $A$ --- or, equivalently, the set of all elements of $G$ obtained by combining finitely many elements of $A$.  (Note that the closure of the empty set is the trivial subgroup.)
\item The \emph{generating set-rank} or \emph{rank} of a group $G$, denoted $d(G)$ or $r(G)$, is the smallest $n$ such that there exists an $A\subseteq G$ of size $n$ such that $\gen{A} = G$.

\item A \emph{homomorphism} is a function $\varphi$ from a group $G$ to a group $H$ such that $\varphi(xy) = \varphi(x)\varphi(y)$ for all $x,y\in G$.
\item The \emph{trivial homomorphism} from a group $G$ to a group $H$ maps every element of $G$ to the identity of $H$.
\item The \emph{kernel} of a homomorphism $\varphi\colon G\to H$, denoted $\ker\varphi$, is $\{g\in G : \varphi(g) = 1\}$.
\item An \emph{isomorphism} is a bijective homomorphism.  Two groups $G$ and $H$ are \emph{isomorphic}, denoted $G\cong H$, iff there exists an isomorphism between them.
\item An \emph{automorphism} is an isomorphism from a group to itself.

\item Given a nonempty subset $A$ of a group $G$, the \emph{centralizer} of $A$ in $G$ (a subgroup of $G$), denoted $C_G(A)$, is $\{g\in G : (\forall a\in A)(gag^{-1} = a)\}$, i.e., the set of all elements of $G$ that commute with every element of $A$.
\item The \emph{center} of a group $G$, denoted $Z(G)$, is $C_G(G)$.
\item Given a nonempty subset $A$ of a group $G$, the \emph{normalizer} of $A$ in $G$ (a subgroup of $G$), denoted $N_G(A)$, is $\{g\in G : gAg^{-1} = A\}$.
\item A group is \emph{simple} iff it is not the trivial group and its only normal subgroups are the trivial subgroup and itself.
\item A group $G$ is \emph{solvable} iff there exists a chain of subgroups $$1 = G_0\normaleq G_1\normaleq G_2\normaleq\cdots\normaleq G_s = G$$ such that $G_{i+1}/G_i$ is abelian for all $i\in\N_s$.
\item Given a group $G$ and a subgroup $H\leq G$, a \emph{left coset} of $H$ is a set $gH$ for some $g\in G$, and a \emph{right coset} of $H$ is a set $Hg$ for some $g\in G$.

\item The \emph{upper central series} of a group $G$ is the infinite chain of subgroups $Z_0(G)\leq Z_1(G)\leq Z_2(G)\leq \cdots$ defined by:
 \begin{itemize}
 \item $Z_0(G) = 1$
 \item $Z_1(G) = Z(G)$
 \item $Z_{i+1}(G)$ is the unique subgroup of $G$ such that $Z_i(G)\leq Z_{i+1}(G)$ and $Z_{i+1}(G)/Z_i(G) = Z(G/Z_i(G))$.
 \end{itemize}
\item The \emph{lower central series} of a group $G$ is the infinite chain of subgroups $G^0\geq G^1\geq G^2\geq \cdots$ defined by:
 \begin{itemize}
 \item $G^0 = G$
 \item $G^1 = [G,G]$
 \item $G^{i+1} = [G, G^i]$
 \end{itemize}
\item A group $G$ is \emph{nilpotent} iff there exists $c\in\N$ such that $Z_c(G) = G$ --- or, equivalently, such that $G^c = 1$ --- in which case the smallest such $c$ is the \emph{nilpotence class} of $G$.

\item Given elements $x,y$ of a group $G$, the \emph{commutator of $x$ and $y$}, denoted $[x,y]$, is $x^{-1}y^{-1}xy$.
\item Given nonempty subsets $A,B$ of a group $G$, $[A,B] \coloneqq \gen{\{[a,b] : a\in A, b\in B\}}$.
\item The \emph{commutator subgroup} of a group $G$, denoted $G'$, is $[G,G]$.
\item The \emph{conjugacy classes} of a group $G$ are all of the subsets of $G$ of the form $\{gxg^{-1} : g\in G\}$ for some $x\in G$.
\item The \emph{exponent} of a group $G$ is the smallest positive integer $n$ such that $x^n = 1$ for all $x\in G$, or $\infty$ if there is no such positive integer.

\end{itemize}


\section{Selected Group Families}
\begin{itemize}
\item The \emph{trivial group}, denoted $1$, is the group containing one element with a trivially-defined operation.

\item Given a positive integer $n$, the \emph{cyclic group of order $n$}, denoted $\Z_n$, is the set $\N_n$ with the operation defined by: $$x\cdot y\coloneqq (x+y)\pmod{n}$$  It has the following properties:
 \begin{itemize}
 \item The identity is 0.
 \item The inverse of an $x\in\Z_n$ is $-x\pmod{n}$.
 \item The order of $\Z_n$ is $n$.
 \item The order of an $x\in\Z_n$ is $n/\gcd(x,n)$.
 \item $\Z_n$ is abelian.
 \item $\Z_n$ represents addition \emph{modulo} $n$.
 \item Every nontrivial finite abelian group can be uniquely expressed as $\Z_{n_1}\times\Z_{n_2}\times\cdots\times\Z_{n_s}$ for some integers $n_1, n_2, \ldots, n_s$ such that each $n_i$ is greater than 1 and is divisible by the $n_{i+1}$ after it (if any).  These integers $n_i$ are the \emph{invariant factors} of the group.
 \end{itemize}

\item Given a positive integer $n$, the \emph{multiplicative group of integers \emph{modulo} $n$}, denoted $\Z_n^\times$, is the set $\{i\in\N_n : \gcd(i,n) = 1\}$ with the operation defined by: $$x\cdot y\coloneqq xy\pmod{n}$$  It has the following properties:
 \begin{itemize}
 \item The identity is 1.
 \item The inverse of an $x\in\Z_n^\times$ is $x^{\varphi(n)-1}\pmod{n}$, where $\varphi$ is Euler's totient function.
 \item The order of $\Z_n^\times$ is $\varphi(n)$.
 \item $\Z_n^\times$ is abelian.
 \item $\Z_n^\times$ represents multiplication \emph{modulo} $n$.
 \item $\Z_n^\times \cong \Aut(\Z_n)$
 \end{itemize}

\item Given groups $G$ and $H$, the \emph{direct product of $G$ and $H$}, denoted $G\times H$, is the set $G\times H$ with the operation defined by: $$(g_1, h_1)\cdot(g_2, h_2)\coloneqq (g_1g_2, h_1h_2)$$  It has the following properties:
 \begin{itemize}
 \item The identity is $(1,1)$.
 \item The inverse of an $(x,y)\in G\times H$ is $(x^{-1}, y^{-1})$.
 \item The order of $G\times H$ is $|G| |H|$.
 \item The order of an $(x,y)\in G\times H$ is $\lcm(|x|, |y|)$.
 \item $G\times H$ is abelian if \& only if both $G$ and $H$ are abelian.
 \item $G\times H \cong G\rtimes_\varphi H$ where $\varphi$ is the trivial homomorphism.
 \end{itemize}

\item Given groups $G$ and $H$ and a homomorphism $\varphi\colon H\to\Aut(G)$, the \emph{semidirect product of $G$ and $H$ with respect to $\varphi$}, denoted $G\rtimes_\varphi H$ or $G\rtimes H$, is the set $G\times H$ with the operation defined by: $$(g_1, h_1)(g_2, h_2) \coloneqq (g_1\varphi(h_1)(g_2), h_1h_2)$$  It has the following properties:
 \begin{itemize}
 \item The identity is $(1,1)$.
 \item The inverse of a $(g,h)\in G\rtimes H$ is $(\varphi(h^{-1})(g^{-1}), h^{-1})$.
 \item The order of $G\rtimes H$ is $|G| |H|$.
 \item $G\rtimes H$ is abelian if \& only if both $G$ and $H$ are abelian and $\varphi$ is the trivial homomorphism.
 \item For all $h\in H$ and $k\in K$, $(1,k)(h,1)(1, k^{-1}) = (\varphi(k)(h), 1)$.
 \end{itemize}

\item The \emph{Klein 4-group}, denoted $V_4$, is $\Z_2\times\Z_2$.
\item Given a prime $p$ and positive integer $n$, the \emph{elementary abelian group of order $p^n$}, denoted $E_{p^n}$, is $\underbrace{\Z_p\times\Z_p\times\cdots\times\Z_p}_{n\ \mathrm{times}}$.
\item The \emph{automorphism group} of a group $G$, denoted $\Aut(G)$, is the set of all automorphisms of $G$ with function composition as the operation.
\item The \emph{holomorph} of a group $G$, denoted $\operatorname{Hol}(G)$, is $G\rtimes_\varphi\Aut(G)$, where $\varphi$ is the identity function.

\item Given a nonnegative integer $n$, the \emph{symmetric group of degree $n$}, denoted $S_n$, is the set of all permutations on a set of $n$ elements with function composition as the operation.  It has the following properties:
 \begin{itemize}
 \item The identity is the identity function.
 \item The inverse of a $\sigma\in S_n$ is the inverse function of $\sigma$.
 \item The order of $S_n$ is $n!$.
 \item $S_n$ is abelian if \& only if $n<3$.
 \item $S_0 = S_1 = 1$
 \item Given nonnegative integers $n\leq m$, $S_n \leq S_m$.
 \item Every finite group is isomorphic to a subgroup of some $S_n$.
 \end{itemize}
\item Given a nonnegative integer $n$, the \emph{alternating group of degree $n$}, denoted $A_n$, is the set of all even permutations on a set of $n$ elements with function composition as the operation.  It has the following properties:
 \begin{itemize}
 \item The identity is the identity function.
 \item The inverse of a $\sigma\in A_n$ is the inverse function of $\sigma$.
 \item The order of $A_n$ is $1$ when $n<2$ and $n!/2$ otherwise.
 \item $A_n$ is abelian if \& only if $n<4$.
 \item $A_n \leq S_n$
 \item $A_n = S_n$ if \& only if $n$ is 0 or 1.
 \item $A_0 = A_1 = A_2 = 1$
 \item Given nonnegative integers $n\leq m$, $A_n \leq A_m$.
 \end{itemize}
\item A \emph{permutation group} is a subgroup of some $S_n$.

\item Given a positive integer $n$, the \emph{dihedral group of order $2n$}, denoted $\Dih_n$, is the set $\N_2\times\N_n$ with the operation defined by: $$(s_1, r_1)\cdot(s_2, r_2)\coloneqq \begin{cases}(s_1, (r_2 + r_1)\pmod{n}) & s_2 = 0 \\ (1-s_1, (r_2 - r_1)\pmod{n}) & s_2 = 1\end{cases}$$  It has the following properties:
 \begin{itemize}
 \item The identity is $(0,0)$.
 \item The inverse of an $(s,r)\in\Dih_n$ is $(s,r)$ itself when $s=1$ and $(0, -r\pmod{n})$ otherwise.
 \item The order of $\Dih_n$ is $2n$.
 \item The order of an $(s,r)\in\Dih_n$ is 2 when $s=1$ and $n/\gcd(r,n)$ otherwise.
 \item $\Dih_n$ is abelian if \& only if $n<3$.
 \item $\Dih_n$ represents the symmetries of a regular $n$-gon; $(1,0)$ is reflection, and $(0,1)$ is rotation by $2\pi/n$ radians.
 \item $\Dih_n \cong \Dih(\Z_n) \cong \Z_n\rtimes\Z_2$
 \end{itemize}
\item Given an abelian group $G$, the \emph{generalized dihedral group of $G$}, denoted $\Dih(G)$, is $G\rtimes_\varphi\Z_2$ where $\varphi$ maps the nonidentity element of $\Z_2$ to inversion on $G$.

\item Given a group $G$ and a normal subgroup $N\normaleq G$, the \emph{quotient group} $G/N$ is the set $\{gN : g\in G\}$ with the operation defined by: $$g_1N \cdot g_2N \coloneqq (g_1g_2)N$$  It has the following properties:
 \begin{itemize}
 \item The identity is $1N$.
 \item The inverse of a $gN\in G/N$ is $(g^{-1})N$.
 \item The order of $G/N$ is $|G|/|N|$.
 \end{itemize}

\item Given an integer $n$ greater than 1, the \emph{dicyclic group of order $4n$}, denoted $\Dic_n$, is the set $\N_{2n}\times\N_2$ with the operation defined by: $$(i_1, j_1)\cdot(i_2, j_2)\coloneqq \begin{cases}((i_1 + i_2) \pmod{2n}, j_2) & j_1 = 0 \\ ((i_1 - i_2 + nj_2) \pmod{2n}, 1-j_2) & j_1 = 1\end{cases}$$  It has the following propertes:
 \begin{itemize}
 \item The identity is $(0,0)$.
 \item The inverse of an $(i,j)\in\Dic_n$ is $(-i\pmod{2n}, j)$ when $j=0$ and $((i+n)\pmod{2n}, j)$ when $j=1$.
 \item The order of $\Dic_n$ is $4n$.
 \item The order of an $(i,j)\in\Dic_n$ is $2n/\gcd(i,2n)$ when $j=0$ and 4 when $j=1$.
 \item $\Dic_n$ is not abelian.
 \item $\Dic_n$ can alternatively be viewed as $\gen{\{i,j\}}$ where $|i| = 2n$, $|j| = 4$, $j^{-1}ij = i^{-1}$, and $i^n = j^2$.  This last value is often denoted $-1$, with $(-1)x$ abbreviated to $-x$ for all $x\in\Dic_n$.
 \end{itemize}
\item Given an integer $n$ greater than 1, the \emph{generalized quaternion group of order $2^{n+1}$}, denoted $Q_{2^{n+1}}$, is $\Dic_{2^{n-1}}$.
\item The \emph{quaternion group}, denoted $Q_8$, is $\Dic_2$.  Its elements are often written as follows:
 \begin{itemize}
 \item $(0,0)$ --- $1$
 \item $(1,0)$ --- $i$
 \item $(0,1)$ --- $j$
 \item $(1,1)$ --- $k$
 \item $(2,0)$ --- $-1$
 \item $(3,0)$ --- $-i$
 \item $(2,1)$ --- $-j$
 \item $(3,1)$ --- $-k$
 \end{itemize}
 It has the following properties:
 \begin{itemize}
 \item The identity is 1.
 \item The inverse of 1 is 1, and the inverse of $-1$ is $-1$.  The inverse of any other element $x$ is $-x$.
 \item The order of $Q_8$ is $8$.
 \item The order of 1 is 1, and the order of $-1$ is 2.  The order of any other element is 4.
 \item $Q_8$ is not abelian.
 \item $i^2 = j^2 = k^2 = -1$
 \item $ij = k$
 \item $jk = i$
 \item $ki = j$
 \item $ji = -k$
 \item $kj = -i$
 \item $ik = -j$
 \end{itemize}
 Note the similarities to cross products of vectors in $\mathbb{R}^3$.

\end{itemize}

\end{document}
